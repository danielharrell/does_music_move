\documentclass[12pt]{memoir}
\title{Does Music Move?}
\author{}   
\date{}
\begin{document}
\maketitle

\subsection{Introduction}

Does music move?

Well, yes---if by ``music'', we meant the vibrations of sound that allow
us to hear what a musician plays: from instrument to air to ear.

Or yes again---if by ``move'', we meant the way we can indeed be
\emph{moved} by music: from table-taps to tangos to tears.

But the question becomes harder to answer, if we ask it of music and
movement in a more elementary sense. ---The sense in which we might say,
of a rhythm, that it quickens and slows; or of a melody, that it rises
and falls; or of a harmony, that it departs and returns. And suppose we
say all this about the first movement of a symphony, not thinking twice
about calling this a ``movement''. For we talk as if we hear just that
in the music---movement---and as if any piece of music indeed moves
itself in moving us.

The talk has a point. For if we didn't hear music move, would we hear it
at all? Without movement, music would seem no more than a series of
sounds.\footnote{For further discussion of this, see Chapter VII, ``The
  Paradox of Tonal Motion'', in Victor Zuckerkandl's \emph{Sound and
  Symbol: Music and the External World}; and the section ``Movement'' of
  Chapter 2, ``Tone'', in Roger Scruton's \emph{The Aesthetics of
  Music}.} But there is a problem with the talk, despite its point. And
this is the problem I discuss in my lecture tonight. In its first
part, I explain what I take the problem to be. And in
its second part, I explain why I take the problem to be important---even for
those of us with little interest in music.

\subsection{The nature of the problem.}

So what is the problem with our talking as if music moved? The problem,
in a word, is \emph{space}: the space in which music moves, when we hear
it as music. ---The space that allows it to quicken and slow, rise and
fall, depart and return. For this space makes something close to
complete sense on the one hand, yet something closer to
nonsense on the other.\footnote{For a defense of the
  necessity of thinking that music moves in space, but that the space in
  question is metaphorical, see Chapter 4, ``Movement'', in Roger
  Scruton's \emph{Understanding Music}.}

\subsubsection{The sense it makes.}

To see the sense it makes, we can start by comparing the movement we
hear in music, with the variation we hear in sounds more generally.
Consider, for example, the difference between a melody and a siren. Both
involve a change in pitch over time. And we might say, in that respect,
that both rise and fall. But unlike the siren, the melody does this in a
kind of articulated space, usually conceived as a scale. And this space
gives us the impression of movement, from one place to the next. We
sense a change of place, in other words, \emph{within} the melody's
change in pitch, as if having caused it.

For example, suppose we hear the beginning of ``Twinkle Twinkle, Little
Star'' on middle-C, where C is followed by the G a fifth above. Within
this change in pitch, we hear a change of place---from the first degree
of a melodic scale to the fifth; or otherwise put, from the $\hat{1}$ to
the $\hat{5}$. And this change of place gives us the impression of a
movement having caused the change in pitch: in this case, of a leap from
the $\hat{1}$ to the $\hat{5}$, causing C to be followed by G.

Now suppose we hear the siren rise from C to G. We don't hear this rise
happen in the articulated space of a scale; and indeed the continuity of
the rise would seem to preclude it, since the places in a scale are
discrete. But this means we are given no impression of movement by the
siren's change in pitch, from one place to the next. It's as if the
siren rises only in time, not in space. True, we can see, or at least
infer, a change of place \emph{behind} the siren's change in pitch, in,
say, the fire engine that produced it. But we don't hear any change of
place \emph{within} the change in pitch, that might replace the fire
engine as cause, and turn what we hear as a signal into what we could
hear as a melody.

Of course, we may not know about a musical scale, to account for the
change of place we hear in the melody. But this ignorance is only more
evidence for the sense made by the space of music's movement, in its
\emph{own} terms. And it reflects a striking fact about what it means to
understand music. For we can develop an altogether discerning musical
ear, while remaining all but illiterate about what we hear.

And becoming literate, by studying music theory, underlines the sense
music makes even without this theory. For the topography of this
theory---such as the melodic scale---is more discovered than invented,
and in the discovery, more inhabited than observed. One sign of this is
the way music moves us; for we are thus moved inwardly, which suggests
that the mode of music's movement is similarly inward. But I will now
try to show this more explicitly. And my conclusion will be that it is
because we seem to \emph{inhabit} what we hear, in hearing music move,
that the space of this movement makes complete sense. The completeness
of this sense has to do with the experience of inhabitance.

To begin to see this, let us take a closer look at the terms I used for
the melody ``Twinkle, Twinkle.'' Consider, first, the way I identified
its first two pitches by note---middle-C and the G a fifth above. This
identification depends on two topographical facts in our perception of
notes, even outside a musical context. The first such fact is that we
hear a difference in pitch, whereby one note is distinguished from
another, as a difference in relative place: one note sounding higher,
the other note lower. This perception is also transitive: if one note
sounds higher than a second, and the second higher than a third, then
the first will also sound higher than the third. Our perception of pitch
difference so gives every note its own position along an axis of height.
Hence my talk of the G a fifth \emph{above} middle-C, to distinguish it,
say, from the C a fourth below.

Then there is my reference just now to the two G's on either side of
middle-C, along with my reference to middle-C itself, in order to
distinguish that C from every other C there is. This repetition of note
letters reflects the second topographical fact in our perception of
notes: the phenomenon of the octave. For while we distinguish notes by
their difference in pitch, this difference reaches a kind of limit at
the interval called the ``octave.'' At this interval, the notes sound
the same, despite their difference in pitch, and are thus given the same
letter as name. Exactly \emph{why} we hear this sameness is hard to say:
Aristotle attributed it to a perception of the whole number ratio
two-to-one; while Victor Zuckerkandl deems it a miracle.\footnote{Aristotle
  attributes the cause of the octave to the ratio 2:1 in Book II,
  Chapter 3, of his \emph{Physics}, 194b28--29. Zuckerkandl calls it a
  miracle in his discussion of the octave in the section ``Scale'' of
  Chapter VIII, ``The True Motion of Tones'', in \emph{Sound and
  Symbol}.} But however it happens, the sameness we hear in notes, once
their difference in pitch reaches the octave, in effect contains that
difference. If we pass beyond it, we don't encounter new notes, but only
new instances, higher or lower, of old notes. The octave thus turns the
axis of height, along which notes are arranged by pitch, into a kind of
circumference, which continues to trace their increase or decrease in
pitch without end, but always to the same place again.

Yet this image, of notes now arranged into a circumference by the
octave, is not yet an image of the space in which music moves. For it
only comprehends the change in pitch involved, as we might conceive this
change to carry us \emph{along} the circumference. But in hearing a
melody, we again hear not only a change in pitch, but a change of place
\emph{within} the change in pitch, such as the leap from the $\hat{1}$ to
the $\hat{5}$ at the start of ``Twinkle, Twinkle''. And this description
reflects another pair of topographical facts in our perception of notes,
once we hear these notes in a musical context. The first such fact is
that we hear movement from note to note, without having to hear any
notes between. So in ``Twinkle, Twinkle'', we hear a leap from the
$\hat{1}$ to the $\hat{5}$, without having to hear the $\hat{2}$, $\hat{3}$, and
$\hat{4}$ first. And to hear them first would not complete the leap, as
if to fill it in, but rather transform the leap into a climb. Why this
is a fact of musical perception may be as hard to explain as the octave.
But as a phenomenon, it seems to involve a sense of being
\emph{oriented} in the movement we hear from note to note. It is as if
we faced the note we were moving to, and reached it as a goal, heedless
of any notes on the way.

The sense of being oriented among notes is even clearer in the second
topographical fact, which involves our perception of notes in at least
those musical contexts we call \emph{tonal}. And it is this fact that
forced me to shift from letters to numbers, when identifying the leap in
``Twinkle, Twinkle'' from the $\hat{1}$ to the $\hat{5}$. For in tonal
musical contexts---and the name ``tonal'' is derived from this fact---we
will hear one note as a kind of center, which \emph{orients} us with
respect to the other notes we hear, as if to provide a place from which
to face them. The central note is accordingly assigned the number
``1'' in analysis, and the other notes assigned other numbers in
reference to ``1''. Again, why we are able to hear a certain note as a
center for other notes is hard to say. But the phenomenon, as
Zuckerkandl would remind us, is dynamic, in an orientation more felt
than seen. We hear the central note as central, that is, by sensing a
stability in it relative to the other notes, as if it provided a place
to face them as a center of gravity. This second topographical fact so
informs the first: for we then move from note to note as if under the
influence of a gravitational pull, requiring effort to overcome, and
supplying momentum in success, in a deepened sense of having faced the
note being moved to, and reaching that note as a goal.

If we accommodated these facts of musical perception into the earlier
image of a circumference of notes, arranged by pitch and bounded by
octave, then we might describe it like this: we hear the movement in a
melody, as in music more generally, as if we were \emph{inside} that
circumference. For once we hear one note in the melody as central,
especially if we hear it as a center of gravity, it is as if that note
has been projected \emph{inside} this circumference from its place along
it. And this projection allows us to move from note to note not simply
\emph{along} the circumference, through every note between, but now
\emph{across} the circumference, guided by the one note inside it as a
center of orientation. And this image so gives a geometric form to the
complete sense made by the space in which music moves. For this is the
sense in which we \emph{inhabit} that space.

To be sure, the development of this sense, as shown by the specifically
\emph{tonal} context in which we hear one note of a melody as central,
depends on the like development of a specific form of musical art---the
art of tonality, to which we owe the music of the West. But there would
be no such art to develop, unless the result made a difference to what
we could hear; and in this case, to what we could inhabit in what we
hear. And much of tonal music's development can be explained as a
deduction from the features of a place we inhabit.

I spell this thought out briefly, in one example, for those familiar
with this development. If we inhabited the space of music's movement, it
seems we should not be simply fixed at a single center of orientation.
We should rather be able move that center, carrying it with us from
place to place. And move it we can---once tonal music developed the
device of modulation, to carry us from key to key. We might also expect
the movement of that center to happen along an axis of depth,
away-and-back, not just to distinguish it from the up-and-down movement
between notes along an axis of height, but also from a sense of
perspective, which implicitly belongs to our sense of orientation when
we inhabit a place. And so we do move away and back---once tonal music
developed the harmony out of polyphony that modulation relies upon. We
might further expect the movement of the center to clarify our sense, as
I described this above, of the gravity felt at work in such a space. And
clarified it is---once the use of modulation effectively reduced the
modes of chant to the major-minor scale. We might finally expect the
movement of the center to deepen our sense that we inhabit \emph{one}
space, which \emph{contains} the places moved between, rather than many
spaces distinguished and divided by those places. And deepened it
is---once the use of modulation forced upon the tuning of a scale the
leveling of equal temperament, restricting the notes of \emph{any} key
to the notes in \emph{every} key, and thereby placing ``the whole of
tonal space,'' as Roger Scruton has strikingly put it, ``within reach of
its every occupant.''\footnote{Roger Scruton, \emph{The Aesthetics of
  Music}, p.~244.}

Along with this deduction of tonal music's development from the features
of a space we inhabit, comes a plausible if simple measure of the
greatness in a tonal musical work. The greater it is, the deeper it
carries us into the space of music's movement. And by that measure of
inhabitance, the best demonstration of the complete sense this space can
make, is found not in the account I have just given of it, but rather in
those masterpieces of tonal music---such as Beethoven's
\emph{Eroica}---where this space is explored to a kind of limit, in the
conquest of it.

\begin{center}\rule{3in}{0.4pt}\end{center}

But so much, then, for \emph{my} account of the kind of sense this space
finally makes. It is time to show why it also makes a kind of nonsense,
despite my account, and all the masterpieces of tonal music
notwithstanding.

\subsubsection{The nonsense it makes.}

To see this, we can start by now comparing the movement we hear in
music, with motion as we observe this more generally. Consider, for
example, the difference between the movement of a melody, like
``Twinkle, Twinkle'' again, and the motion of our hand in following that
melody, as if to conduct it. We could say that both rise and fall, and
do so not just in time but in space, through a discernible change of
place. But there is a difference. For as I mentioned earlier, we hear
the change of place in the melody, such as the leap from the $\hat{1}$ to
the $\hat{5}$ in ``Twinkle, Twinkle'', without having to hear any
places, or tones, between. The movement happens, so we might say,
\emph{discretely}. By contrast, we see the hand's change of place happen
\emph{continuously}, from place to place through \emph{all} the places
between. And this continuity would seem a necessary feature of its
motion. For if our hand got from place to place discretely, like the
melody did, skipping places along the way, then it would look to us as
if our hand reached each place not by motion, but rather by magic. Or at
least we'd be tempted to think there was something in the space where
the motion occurred, beyond \emph{just space}, that was interrupting the
motion, disrupting the continuity it would otherwise have. But this is
one reason, then, to think there isn't really a space for the melody to
move in. For if there were, then it would allow the melody to move in it
continuously rather than discretely.

But this reflects another difference between the movement of the melody
and the motion of our hand in following it. And this difference involves
the matter of identity rather than continuity. For the melody is
\emph{composed} of the notes it moves between, while our hand is not
composed of the places it moves between. And this explains at once why
the melody has to move discretely. It has to move discretely, because it
has \emph{to become} what it is. And this means passing through only
those notes that compose it, and that distinguish it, thus composed,
from any other melody. But then the melody can only \emph{be} what it is
by becoming so, over an interval of time. And in this sense, the melody
is temporal rather than spatial, with an identity in time rather than
space. But this is then another reason to think that there isn't really
a space for the melody to move in. For if there were, then it would
allow the melody to possess an identity in space; which is to say, it
would allow the melody to \emph{be} what it is in space without having
to become so, and to remain what it is, unchanged, through every change
of place.

But here is perhaps a stronger way to put the nonsense: If there really
were a space for the melody to move in, then there would be a melody to
hear, in the space where we hear it move. But there isn't. We hear the
leap in ``Twinkle, Twinkle'', for example, without hearing anything
making the leap. For we hear this leap being made between unleaping
notes, and hear nothing further to which we might attribute the leap.
And this seems true for music in general. We hear a movement being made
between unmoving notes, and nothing further to which we might attribute
the movement. So we hear movement, but nothing making the movement. Yet
how, in that case, could there \emph{be} any movement to hear? And what
could be making it?

Yet this is only the start of the nonsense. And the end of it implicates
our very inhabitance of the space in question. We can see this through
an objection to the analysis I just gave. True, the objection runs, we
have a sense of the space in which motion ordinarily occurs, that we
cannot apply to the movement of music. For applying it makes nonsense of
the movement, by depriving this movement of any object to which it might
be attributed. But what follows from this? Perhaps only that the space
in which music moves, is not ordinary, but extraordinary. And this is
one way to understand my earlier defense of the sense---the complete
sense---made by this space. For this was not an observed sense of space,
but rather an inhabited sense of it. The sense, for example, in which we
\emph{face} the note being moved to, and reach it as a goal. So if there
\emph{is} a space for music to move in, which makes sense of the
movement, then this space will have to be conceived from within, as a
matter of inhabitance, rather than from without, as a matter of
observation.

Well, fair enough. But this makes the nonsense of such a space even
clearer. For what is it, finally, that makes an inhabited sense of space
extraordinary rather than ordinary? Here is one answer---the answer, for
example, that Heidegger might give.\footnote{For a further account of
  Heidegger's actual answer, see Part One, Division One, Chapters II and
  III of his \emph{Being and Time}.} An inhabited sense of space is
extraordinary, in not stopping short of totality. That is, our inhabited
sense of space implicitly includes everything there is, known or
unknown; anything captured in the word ``Being''. For it is beings,
finally, that we take ourselves to be surrounded by, and our own being
that provides the place from which to face them. This is why, despite
our sometimes talking as if there were more than one world, and even
more than one world we might inhabit, we can also talk intelligibly, in
a simpler yet deeper way, of \emph{the} world, as if there were only
one. And it is our being-in-the-world, on this answer, that grounds our
inhabited sense of space.

But what happens, then, when our inhabited sense of space is divided?
---For example, between sleeping and dreaming, where it seems we inhabit
two spaces at once? We resolve the division, evidently, by conferring
worldhood on only one such space, taking it \emph{to contain} the other
such space. That is, we credit only one such space with the totality,
and thus the reality, of inhabitance, and regard the other space as
merely part of this totality. The credit we give to its own totality,
then, is the credit we give to a dream. We still inhabit the
dream---indeed more alertly than we inhabit the bedroom in which we
dream, and often with a sense that everything is put at stake in the
dream, in a matter of life or death. But in waking from the dream, we
prove it to be part of a larger space of inhabitance. And this gives the
dream's apparent totality the status of mere appearance; and our
inhabitance of it, the form of an illusion. To our relief, or perhaps
our regret, what happened to us in the dream, didn't really happen after
all.

But if this is so, then it gives us reason to suspect the very same
thing of the space we inhabit when hearing music move. The space may
well be illusory, making all the sense---but also all the nonsense---of
a dream. And in that case, the greater the work of music, the deeper it
carries us into the dream. We hear the sound of Beethoven's
\emph{Eroica} surge forth in a concert hall, seeming to make the
whole world shake. Yet the musicians barely move by comparison, while the
notes they play move not at all. And we concert-goers stay glued to our
seats---entranced. And once the work is finished, in a triumph of
conclusiveness, we are released from the trance in a daze---and the
desire, perhaps, to have remained. For we leave the concert hall likely
finding the world we truly inhabit unchanged by what we heard, and
nothing comparable to its conclusiveness in the life we have to live.

This, then, is what I take to be the problem with our talking as if
music moved. Talking that way fails to distinguish our experience of
music from a dream, in which nothing we experience really happens.

\subsection{The importance of the problem.}

As I said at the start of my lecture, I also take this problem to be
important. To explain why, I discuss this problem again, but now as a
problem not simply with our experience of music, but more generally,
with our experience of the world. For we talk not just as if music
moved, but as if \emph{anything} moved. Yet this too involves a kind of
nonsense, known since the time of Parmenides. And it proves to be the
same kind of nonsense we encountered in the movement of music.

To see this, recall my earlier analysis of why there really isn't a
space for a melody to move in. For the melody, in this respect, is
temporal rather than spatial. It is composed of the places, or notes, it
moves between, and can only be what it is by becoming so. So there isn't
a space, strictly speaking, for the melody to move in, with the
continuity or identity of truly spatial things, like our hand in
following that melody.

Fair enough. But while our hand may be spatial in this respect, the
\emph{motion} of our hand is temporal, just like the melody. In a sense,
it too is composed of the places moved between, and can only be what it
is by becoming so. But this implies, on the earlier analysis, that there
isn't really a space for our hand to move in, or indeed for anything to
move in. There is only a space for it to be contained in, and to occupy,
over the course of its motion, at every place composing that motion.
Space, in other words, can only contains the path of the motion, not the
motion itself.

But here again is perhaps a stronger way to put the nonsense. If there
really were a space for our hand to move in, then there would be a hand
to see, in the space where we see it move. But there isn't. All we see
is this motion being made from place to unmoving place, by something
that occupies each such place. We see nothing further, to which we might
attribute \emph{the motion} between places, rather than just \emph{the
occupancy} of those places. So we see motion, but nothing making the
motion. Yet how, in that case, could there \emph{be} any motion to see?
And what could be making it?

One answer, the kind Zeno might give, is that there \emph{isn't} any
motion to see.\footnote{For further discussion of Zeno's arguments
  against motion, see Aristotle's \emph{Physics} Book VI, chapters 2 and
  9, and Book VIII, chapter 8; and Lecture VI, ``The Problem of Infinity
  Considered Historically,'' in Bertrand Russell's \emph{Our Knowledge
  of the External World}.} When we think we see something move, all we
really see is that it occupies different places at different times. And
the appearance of motion in this, is something like the appearance of
motion on a movie screen, or in a flip book, from a succession of
images, each of a single place and time, that happens too rapidly for us
to detect. But this first answer leads to a second answer, found for
example in calculus, where there \emph{is} a motion to
see.\footnote{For a further discussion of this answer, see Lecture V,
  ``The Theory of Continuity,'' and Lecture VII, ``The Positive Theory
  of Infinity,'' in Russell's \emph{Our Knowledge of the External
  World}.} When we see something move on this answer, all we indeed see
is that something occupies different places at different times. Still,
it can be proved that there are more such places---infinitely more, in
fact---than we could ever arrange from one to the next. The places are
many enough, that is, to form a \emph{real} continuum, beyond the mere
appearance of one in a rapid enough succession of places arranged one to
the next. And this makes for a real continuity in the motion we see,
giving it all the reality it needs.

There is a sense, however, in which this second answer misses the point.
For the continuity in question belongs to the motion's path rather than
to the motion itself. And the way this path is proved to form a
continuum---for example, in Dedekind---is by analogy to a line,
undivided by time, where every place upon it is present at
once.\footnote{Dedekind draws the analogy in Chapters II and III of his
  \emph{Continuity and Irrational Numbers}.} But it is only when this
line is divided by time, where only one place upon it is ever present at
once---the point right now, so to speak---that it seems we have the
proper analogy for the continuum of the motion, rather than of just its
path. But how can a line divided by time be proved continuous, if it is
made of only one point at a time?

The second answer so leads to a third answer, more philosophical than
mathematical, and found, for example, in Bergson.\footnote{See, for
  example, Bergson's lecture ``The Perception of Change'' in the
  collection \emph{The Creative Mind}. There Bergson even connects his
  account of motion generally to the movement in a melody. Zuckerkandl
  follows Bergson's lead in his own account of music's movement in
  \emph{Sound and Symbol}.} This answer is distinguished from the first
answer in taking the motion we see to be real, and from the second
answer in taking this reality to involve more than just the continuity
of the motion's path. But the promise of this answer comes at a price:
for it embraces the nonsense that makes the problem a problem. On this
answer, that is, there is indeed nothing to see, in the space where we
see anything move. And this is because the space where we see anything
move, is the space where the motion belongs to us, rather than to
anything we might see outside us. So there \emph{is} a motion to see,
when \emph{we} are the ones making it.

On this answer, in other words, motion occurs in the space we inhabit, rather than in
the space we observe. For in the space we observe, we see only the
continuous path of a motion, occupied at every place by the thing that
moves. But in the space we inhabit, we see the motion itself, which is
not simply continuous, but indivisible. The motion in this indivisible
sense stretches from the beginning of its path to the end in a single
bound, as if the entire path were a single place for the moving thing to
occupy. And we know this indivisibility when moving ourselves. For in
that case, we face the end of our motion at the beginning of it, and
reach the end as a goal, heedless of any places on the way---just
like in the movement of a melody. And the path we trace in reaching that
goal can be divided only by our changing the goal, in a motion now
different from what it was, but again indivisible. This too is like the
movement in a melody; for only further notes can divide the distance
from note to note, but thereby produce a new melody out of the old one,
showing the movement between notes in any one melody to be, indeed,
indivisible.

But what should we make of this answer? It makes the same appeal to our
inhabited sense of space that we encountered before in making sense of
the movement in music, only now to make sense of motion as such. But
does it have the same problem? That is, can we show, or know, that we
aren't dreaming when we see anything move?

The question put this generally might seem absurd. After all, to suppose
we \emph{were} dreaming when seeing anything move, would mean suspecting
something illusory about our inhabitance of \emph{the world}, and not
just of so-called other worlds, such as dreams in the ordinary sense, or
music, that we might plausibly find illusory in their proving
\emph{contained} in the world, mere parts of that totality.

But there is good reason to suspect something close to illusory even
about our inhabitance of the world. And it explains why a question like
``Are we dreaming right now?'' has been raised in philosophy---for
example, by Descartes.\footnote{Descartes raises this question in his
  \emph{First Meditation}. It is also discussed in Plato's
  \emph{Theaetetus}, 158b-d; and mentioned, but also dismissed, in
  Aristotle's \emph{Metaphysics}, 1011a6.} For while we may not be
dreaming when we see anything move, there is arguably nothing in our
inhabitance of the world to prove it. And we lack this proof, so the
argument goes, precisely because we inhabit the world rather than merely
observe it.

To see why this argument is worth considering, consider, first, a
striking fact about optical illusions, such as the look of a stick
half-plunged in water, and seemingly half-bent by it. Our knowing the
stick stays straight does nothing to dispel the illusion. But why? One
answer is that there is nothing \emph{in} the illusion to \emph{tell} us
it's an illusion, and if there were, then the illusion wouldn't be an
illusion. We so remain \emph{in} the illusion, or inhabit it, as a
matter of perception, even if we stand outside the illusion, or merely
observe it, as a matter of knowledge.

Something like this is also at work in dreams, taking ``dreams'' again
in the ordinary sense. For our dreams so often contain implausible or
even impossible events, not to mention the sense in which all its
events, once we wake up, prove unreal. So why did they seem so real in
the dream? Again, one answer is that there is nothing \emph{in} the
dream, no matter how implausible or impossible, to tell us we are
dreaming, and if there were, then we wouldn't be dreaming. So we inhabit
the dream rather than merely observe it. It is only when something in
the dream wakes us up from it, and puts us outside the dream to observe
it, that we can know it for the dream that it is.

We can give this same answer to the question of why our experience of
music would still seem so real, even if we were convinced---say, by my
lecture tonight---that it was only a dream. For there is nothing
\emph{in} the music to tell us so, even if there is some account of the
music to tell us so. And again, this is because we hear music as music
only by inhabiting it, past the point of any account we might give of
it.

This answer, then, points to perhaps the most decisive, if negative,
feature of our inhabitance of the world. There is nothing we might
encounter in the world---motion, for example---to tell us the encounter
is illusory. And if there were, then we wouldn't be having the
encounter. So for all we know, we might be dreaming. Arguing against
this prospect, to be sure, is our sense that this is indeed the world
that we inhabit; our sense, in other words, that we are surrounded by
beings rather than seemings, and that we encounter these beings from a
being rather than seeming of our own. But this sense can never be
decisive. For the cost of inhabiting the world, on this answer, is to be
past the point of knowing whether this inhabitance is only a dream.

This, then, is what I take to be the importance of the problem with our
talking as if music moved. For it illustrates a more general problem,
with our talking as if anything moved. Talking that way fails to
distinguish our experience of the world from a dream, in which nothing
we experience really happens.

And one measure of this problem's importance, as well as one solution to
it in effect, is found in how we talk when \emph{explaining} the world
we experience. For then we talk as if our inhabitance of the world is
indeed a dream, of a kind there is no waking up from. Or to put this
Cartesian legacy of scientific discourse a different way: the terms in
which we explain the world, as a matter of observation, are not the
terms in which we encounter the world, as a matter of inhabitance---and
in many cases could \emph{never} be the terms of such an encounter, 
as in quantum mechanics. Yet we take this discrepancy, still, as a sign that the
world has been explained rather than erased. This is why we look
for an explanation of our encounter with music, not in Beethoven's
\emph{Eroica}, but rather in the physicist's account of wave phenomena,
or the neuroscientist's account of brain patterns, or the biologist's
account of evolutionary adaptations, or even the Nietzschean's account
of a sickness in the soul. Despite the fact we don't encounter
music as music in such terms, we credit these accounts as attempts to
explain that encounter. And why? Because we find nothing self-justifying
in the encounter itself: it might well be only a dream. And to solve
this problem, we explain the dream away. That is, we explain the
encounter by avoiding any appeal to the sense the encounter makes only
from a place of inhabitance---and often by challenging that sense.

\subsection{Conclusion}

There is, however, a different solution to the problem. And I conclude
my lecture by presenting this solution in brief. We might call it
Heideggerian to distinguish it from the Cartesian solution just
mentioned. In this solution, not having a proof that our inhabitance of
the world is real, is all the proof for it that we need. There is
nothing to tell us we might dreaming, true; but this is because we
aren't.

This solution already sounds unpromising. After all, there is nothing in
\emph{a dream} to tell us we are dreaming, but we are. On this solution,
however, it turns out we aren't, at least as a matter of inhabitance.
For any space of inhabitance is real, on this account, precisely because
it can be inhabited. And this means we are \emph{always} surrounded by
beings rather than seemings, and always encounter them from a being
rather than seeming of our own, whether we find ourselves in a dream, in
a symphony, or in the world that contains them. True, we have reason to
think otherwise, again since these smaller spaces are contained in a
larger space that proves their own totality to be illusory. But this
assumes that any space of inhabitance lies, indeed, in space. Yet on
this solution, it is only a space of observation that lies in space. Any
space of inhabitance lies instead in time.

Perhaps this solution now sounds even more unpromising. For how can
there be any room in time for a space of inhabitance, when the only part
of time that ever exists is `right now', that instant between the
no-longer and not-yet at the seeming size of a point? Still, this very
fact about time's evanescence turns the illusion of motion into an
object of wonder if not astonishment. For according to our earlier
analysis, motion is what it is, only in time. But if time exists only at
a point, then how could we ever encounter a temporal whole like
motion---even in the form of an illusion? How could we even dream that
we hear the rise and fall of a melody, or see the rise and fall of our
hand in following it?

It is in offering an answer to this question that the solution starts to
look more promising. And the answer is this. `Right now' may be no
larger than a point. But this is still room enough to establish a space
of inhabitance, in providing a center of orientation. And this is the
center from which the not-yet lies ahead of us, the no-longer lies
behind us, and the right-now is always with us. And in being always at
this center, we are always in the world, even when we find ourselves in
a dream, or a symphony. We inhabit even these smaller spaces from the
same center of orientation, where the not-yet is ahead of us as a matter
of expectation, the no-longer is behind us as a matter of memory, and
the right-now is always with us as a matter of attention. It is in these
terms, temporal terms, that any space of observation becomes a place of
inhabitance.

It is also in these temporal terms that any place of inhabitance is real
rather than illusory. For this center of orientation, despite being a
mere point in time, comprehends a totality in time for each of us---from
a beginning in birth that no one else can share, to an end in death that
no one else can know. This is the center, then, from which the not-yet
ahead of us is a future to face; the no-longer behind us is a past to
bear; and the right-now always with us is a present at stake. So in
being always at this center, at a point of totality as it were, we are
always in a world that is real, even when we find ourselves in a dream
or symphony. We inhabit even these smaller spaces from the same center
of orientation, with a future to face, a past to bear, and a present at
stake. This is why we can be released from their spell, without having
recovered from it. And in this sense, again temporal rather than
spatial, we are not carried from the world by a dream or symphony, but
concentrated in it.

Or to summarize this solution to the problem in conclusion: we can only
talk as if \emph{anything} moved, much less as if music moved, insofar
as the space we inhabit in talking that way, makes complete sense of
what we say, without any room for nonsense. And this is the space we
inhabit right now: the space made possible by time.\footnote{This is a
  lecture I gave at the Annapolis campus of St.~John's College on
  February 15, 2013. I would like to thank Gabriela Hopkins for helping
  me improve the final draft with comments and conversation about
  earlier drafts. ---Daniel Harrell}
  
  \end{document}
